\chapter*{Introduction}
\markboth{Introduction}{}
\addtotoc{Introduction}

The concept of symmetries in physics is based on the supposed invariance of physical systems under certain transformations. Whereas symmetries under continuous transformations such as translation in space are of great consequences as they give origin to conservation laws, another extremely important class of symmetry-related transformations is constituted by operations commonly referred to as inversions, which yield the original system when applied to the system twice. The three most relevant inversions are parity, charge conjugation and reversal in time%
\footnote{%
  While the term \textit{time reversal} is commonly used in literature, it has been argued that since the reversal of time as such is clearly unphysical, it is more appropriate to speak of \textit{reversal in time} (the term proposed by John S. Bell) or of \textit{motion reversal} (as originally used by E.~Wigner)~\cite{bernabeu_colloquium, sozzi} In this work, the names \textit{reversal in time} or just \textit{T} will be adopted for the operation and its corresponding symmetry.
}.

Parity operation inverts spatial coordinates with respect to the origin, charge conjugation is an exchange of particles with their antiparticles which makes all charges change sign, and reversal in time applied to a system transforms it to a system with the opposite sense of time, where the spatial coordinates remain unchanged whereas vectors of velocity, momentum and spin change sign. For each of these inversions, a corresponding symmetry of Nature may be considered, i.e.\ parity~($\mathcal{P}$) symmetry, $\mathcal{C}$~symmetry and symmetry under time reversal~(\Ts~symmetry).

Since the introduction of the concept of these fundamental discrete symmetries in microscopic systems described by quantum mechanics by E.~Wigner in 1931~\cite{wigner1931}, large efforts have been made to test these symmetries with various systems and interactions. In fact, all of these symmetries have been found to be violated by the weak interactions but while first deviations from $\mathcal{P}$~and $\mathcal{C}$ symmetries were observed already in 1956 and 1958, respectively~\cite{parity_violation, c_violation}, more than half of a century had passed before direct evidence for nonconservation of the $\mathcal{T}$~symmetry could be found in a measurement with the neutral B meson system performed by the BaBar experiment in 2012~\cite{t_violation_babar}.

After the surprising discovery of violation of the combined \CPs~symmetry by Christenson~\textit{et~al.} in 1964~\cite{cp_violation}, the interest in searches for \Ts~noninvariance had increased as time reversal symmetry-violating effects would be expected from the \CPTs~symmetry. However, even though to date multiple experiments have confirmed \CPs~violation and \CPTs~still appears as a good symmetry of Nature (as tested by numerous experiments~\cite{pdg2016}), the evidence for violation of symmetry under reversal in time is still scarce and tests of this symmetry remain a challenging field of elementary particle physics. On the other hand, despite a multitude of tests of \CPs~(including measurements of the violation level) and test of \CPTs~conducted to date, physical systems and interactions exist for which the fundamental discrete symmetries have been hardly investigated. An example of the latter is constituted by purely leptonic systems and their electromagnetic interactions, for which the violation of both \CPs~and \CPTs~was only recently excluded at the precision level of $10^{-3}$~\cite{cpt_positronium, cp_positronium}. The symmetry under reversal in time has never been studied in this sector.

The aim of this Thesis was to prove the feasibility of two new tests of discrete symmetries. The first one is a direct test of the symmetry under reversal in time in transitions of neutral mesons, following a recently proposed concept~\cite{theory-babar,theory:bernabeu-t}. Such a test, feasible in the systems of flavoured neutral mesons with quantum entanglement, is to date the only experimental technique which provided direct evidence on \Ts~violation through a measurement by the BaBar Collaboration using $B^0$ mesons~\cite{t_violation_babar}. To date, the only possible extension of this test to other physical systems can be performed by the KLOE-2 experiment where quantum-entangled neutral K meson pairs are produced in the decay of $\phi$ mesons created in electron-positron collisions. Hence, a large part of the work presented herein was concentrated on providing the tools required to conduct a direct test of symmetry under reversal in time with neutral kaons at the KLOE-2 experimental setup.
As the latter is in the course of collecting data at the time of writing of this Thesis, an analysis was performed with a dataset collected by the KLOE experiment in 2004--2005. Although sensitivity of \Ts~violation measurement results obtained with this data is limited due to statistics, the goal of this work was to devise steps needed to extract certain transitions of neutral kaons between their flavour and CP-definite states from the data taken by the general-purpose KLOE detector and to demonstrate the feasibility of the \Ts~test with a view to its realization with larger amount of data collected by the KLOE-2 experiment. To this end, a complete analysis of the KLOE 2004--2005 data was prepared including determination of the two \Ts-asymmetric observables of the test.

The second experimental search for discrete symmetry violation elaborated on in this work concerned the lightest purely leptonic system constituted by the bound state of an electron and a positron, i.e.\ a positronium atom. A search for non-vanishing expectation values of operators odd under certain symmetries in angular correlations between photons momenta in the decays of ortho-positronium (\ops/) atoms is one of the objectives of the J-PET (Jagiellonian Positron Emission Tomograph) experiment, capable of improving the present $\order{10^{-3}}$ limits on \CPs~and \CPTs~violation in the purely leptonic systems. A~similar approach may be used to perform the first test of the \Ts~symmetry with positronium atoms~\cite{moskal_potential}. Therefore, a part of this Thesis is devoted to a first attempt to identify \ops/$\to 3\gamma$ events in the data collected with the J-PET system. Reconstruction of such decays and the necessary introductory steps to test the discrete symmetries with orto-positronium decay into photons with J-PET are also described.

Although the two experiments concerned in this Thesis are seemingly different as concerning various physical systems, interactions and energy scales as well as employing different strategies of searching for discrete symmetries' violation, their common traits are more than originating from an electron-positron system. Both experimental cases involve decays of neutral particles into final states comprising several photons, reconstructed solely on the basis of the photons' interactions in a calorimetric detector. Such processes, namely $\Kl\to 3\pi^0\to 6\gamma$ in KLOE and \ops/$\to 3\gamma$ in J-PET, are reconstructed using a trilateration-based approach.

%
% TODO: update if structure changes!
%
This Thesis is divided into eight chapters. Chapter~1 comprises the most essential information about the operation of reversal in time, present state of the \Ts~violation searches as well as properties of the system of neutral K mesons used for the direct \Ts~test, whose theoretical principle is described in Chapter~2. Chapter~3, in turn, discusses the scheme of discrete symmetry tests with ortho-positronium decays at J-PET\@. These discussions of the concept of each experiment are followed by descriptions of the KLOE and J-PET experimental systems, given in Chapter~4. The trilateration-based reconstruction of neutral particle decays into photons, constituting a common point of both presented data analyses, is introduced in Chapter~5. Details of the analysis of KLOE data in view of the \Ts~symmetry test, along with obtained results and a discussion of perspectives for a measurement with the KLOE-2 detector are  discussed in Chapter~6. Subsequently, Chapter~7 contains the results of feasibility tests of reconstruction of $e^+e^-$ annihilations into three photons at J-PET in the context of planned searches for non-vanishing angular correlations of photon momenta in these decays. A summary of the results of both studies and an outlook for the capabilities of future measurements based on the presented work follow in the last Chapter. Finally, the text is closed by a set of Appendices comprising details and derivations of several numerical and data analysis methods used in this work.


%%% Local Variables:
%%% TeX-master: "../main"
%%% End: 
