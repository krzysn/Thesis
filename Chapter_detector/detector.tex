\chapter{The HADES detector}
\label{chapter:detector}
The \textbf{H}igh \textbf{A}cceptance \textbf{D}i-\textbf{E}lectron \textbf{S}pectrometer (HADES) \cite{Agakishiev:2009am} is located at the GSI Helmholtzzentrum f{\"u}r Schwerionenforschung. The HADES detector was designed for various measurements with the especial emphasis on di-electron spectroscopy. Thanks to versatility of the SIS18 (German: \textbf{S}chwer\textbf{I}onen\textbf{S}ynchrotron) accelerator and a secondary pion beam facility, various kind of experiments can be conducted: starting from pion scattering on proton or nucleus targets, through proton-proton and proton-nucleus reactions, up to the heavy ion collisions.

The detector provides almost full azimuntal angular coverage, whereas the acceptance in the polar angle used to rate from 18$^{\circ}$  to 80$^{\circ}$. A current upgrade extands the detector acceptance for forwards angles, for more detaisl see \ref{subsec:FwDet}. Two sets of \textbf{M}ulti-wire \textbf{D}rift \textbf{C}hambers (MDC) together with a superconducting toroid magnet allow for momentum measurements with $\frac{dp}{p} \approx 2-3\%$ and particle identification (PID) via energy loss measurement. The PID is further enhanced by high resolution \textbf{T}ime \textbf{O}f \textbf{F}light (TOF) detectors ($\sigma \approx 80$ ps) and a hadron-blind \textbf{R}ing \textbf{I}maging \textbf{CH}erenkov (RICH) detector. A combined information form the detectors allow for efficient p/$\pi$/K/e separation over broad momentum range.
\section{Tracking system}
The HADES tracking system bases on four sets of a drift chambers. Two before and two after a magnetic field. First set is called inner MDC, the second outer MDC. Each single drift chamber has a trapezoidal shape and consist of 13 layers of wires. They create 6 layers of a drift cells. A shape of the sells and the wires dencity was optimalized to get the best momentum resolutions.

Inbetween iner- and outer-MDC the \textbf{I}ron\textbf{L}ess \textbf{S}uperconducting Electron (ILSE) magen is located. It consist of six superconducting coils, which produce a toroidal magnetic field of 3.6 T inside the coils. The operational current can variy from 0 to 3500 A. 

The magnetic field produced by ILSE bends particles' tracks what allows for a momentum reconstruction. Tracks reconstruced in iner- and outer-MDCs have to be matched together. A deticated algorithm for this purpuse was developed by HADES collaboration. 


\section{RICH detector}

\section{Start detector and a trigger system}

\section{The HADES upgrades}

\subsection{The Forward Detector}
\label{subsec:FwDet}
In many studies especially devoted to hyperons' decays the forward angles play very important role. The hades detector for a quite long time did not have a possibility to register particles 
\subsection{RICH update}

\subsection{Electromagnetic calorimeter}

\label{chapter:HADES_upgrades}